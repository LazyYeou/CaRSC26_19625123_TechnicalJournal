\documentclass[11pt, paper=a4, DIV=12]{scrartcl}

\usepackage[utf8]{inputenc}
\usepackage[bahasa]{babel} 
\usepackage{tcolorbox}
\usepackage[margin=2cm]{geometry}
\usepackage{hyperref}
\usepackage{enumitem} 
\usepackage{listings}
\usepackage{xcolor}

\definecolor{codegreen}{rgb}{0.13, 0.55, 0.13}
\definecolor{codegray}{rgb}{0.47, 0.47, 0.47}
\definecolor{codepurple}{rgb}{0.58, 0, 0.82}
\definecolor{backcolour}{rgb}{0.97, 0.97, 0.97} 
\definecolor{codeblue}{rgb}{0.16, 0.32, 0.75}
\definecolor{codered}{rgb}{0.7, 0.11, 0.11}

\lstdefinestyle{inistyle}{
    backgroundcolor=\color{backcolour},   
    commentstyle=\color{codegreen},
    keywordstyle=\bfseries\color{codeblue},
    numberstyle=\tiny\color{codegray},
    stringstyle=\color{codered},
    basicstyle=\ttfamily\small,
    breakatwhitespace=false,         
    breaklines=true,                 
    captionpos=b,                        
    keepspaces=true,                 
    numbers=left,                    
    numbersep=10pt,                      
    showspaces=false,                
    showstringspaces=false,
    showtabs=false,                  
    tabsize=4,                           
    frame=lines,                         
    rulecolor=\color{black!80},          
    framerule=1pt,
    aboveskip=15pt,                      
    belowskip=15pt,                    
    xleftmargin=15pt,                    
}

\lstset{style=inistyle}

\title{Technical Journal RSC Aksantara 2026}
\author{Muhammad Ridwan Nasir Firdaus}
\date{Januari -- Februari 2026}

\begin{document}
\maketitle

\tableofcontents 
\pagebreak

\section{Kamis, 29 Januari 2026} 
\begin{tcolorbox}[colback=blue!5,colframe=blue!75,title=Poin Utama]
\begin{itemize}[noitemsep] 
    \item Pengenalan RSC
    \item Arsitektur Sistem UAV
    \item Konsep OOP 
    \item Error Handling
    \item Multithreading Dasar
\end{itemize}
\end{tcolorbox}

\subsection{Pengenalan RSC}
RSC adalah departemen dalam Aksantara yang mengelola seluruh aspek terkait dengan perangkat lunak (\textit{software}). Dalam RSC terdapat dua keilmuan, yaitu \textit{Control and Perception} (ConCept) dan \textit{Ground Control Station }(GCS).

\subsubsection{Control and Perception}
Berfokus dalam pengembangan sistem navigasi untuk otonom dengan menggunakan algoritma tingkat tinggi berbasis ROS2 dan pengembangan persepsi dengan \textit{Computer Vision}.

\subsubsection{Ground Control Station}
Berfokus dalam pengembangan \textit{software} Ground Control Station internal.

\subsection{Arsitektur Sistem UAV}
\subsubsection{Wahana}
\begin{itemize}
    \item \textbf{Airframe}: Struktur utama drone yang terdiri atas \textit{Fuselage}/badan, ekor, dan sayap untuk FW, sedangkan untuk VTOL terdiri atas \textit{frame center} dan \textit{arm}.
    \item \textbf{Sistem Propulsi}: Terdiri dari motor, propeler, \textit{Electronic Speed Controller}(ESC), dan baterai LiPo.
\end{itemize}

\subsubsection{Avionics}
\begin{itemize}
    \item \textbf{Flight Controller (FC)}: Berperan sebagai saraf motorik dari UAV yang memproses dan menyatukan data dari sensor serta untuk menjalankan algoritma kontrol penerbangan. Dalam FC terdapat komponen lain seperti MCU, barometer, IMU, dan \textit{blackbox}.
    \item \textbf{Modul GNSS}: Menerima sinyal dari sistem satelit seperti GPS, Galileo, dan BeiDou untuk menentukan lokasi, mengukur kecepatan, dan waktu UAV. 
    \item \textbf{Airspeed Sensor}: Menghitung kecepatan relatif wahana terhadap kecepatan udara.
\end{itemize}

\subsubsection{Payload}
Merupakan komponen yang diangkut oleh drone dan tidak esensial untuk keperluan penerbangan. Komponennya seperti kamera fpv, kamera tracking, LiDar, kamera depth, dan sensor partikel.

\subsubsection{Sistem Komunikasi Radio}
Komunikasi antara UAV dengan GCS adalah gelombang radio dalam frekuensi tertentu. Jalur komunikasi fungsional yang terbentuk dari pertukaran data lewat frekuensi disebut dengan \textit{link}. Terdapat dua \textit{link} utama dalam sistem UAV, yaitu:

\begin{itemize}
    \item \textbf{Non-Payload Link}: Saluran utama yang bertanggunang jawab atas pengendalian UAV. Sistem ini terjalin antara \textit{radio tranceiver} GCS dengan UAV.
    \item \textbf{Payload Link}: Saluran untuk mengirim data \textit{payload} antara GCS dengan UAV.
\end{itemize}

\subsubsection{Companion Copmuter (CC)}
CC digunakan untuk menangani masalah komputasi tingkat tinggi yang harus dilakukan secara real-time dan melengkapi FC sebagai sistem saraf motorik dari UAV.

%==================================================

\subsection{Object Oriented Programming}

\subsubsection{Class dan Object}
Class merupakan \textit{blueprint} yang berisi sifat-sifat dari suatu objek dan \textit{Object} adalah bentuk atau implementasi nyata dari suatu class.
\begin{lstlisting}[language=C++, caption=Contoh Penerapan Class dan Object]
#include <iostream>
#include <string>

//inisialisasi class Drone
class Drone {
public:
    std::string name;
    int altitude;
    int speed;
};

int main() {
    Drone drone1; //inisialisi object dari class Drone
    drone1.name = "Drone Suki ygy";
    drone1.altitude = 0;
    drone1.speed = 0; 
}
\end{lstlisting}

\subsubsection{Encapsulated}x
\subsubsection{Inherited}
\subsubsection{Abstraction}
\subsubsection{Polymorphism}

\section{Jumat, 30 Januari 2026}
% Isi catatan berikutnya...
\textit{Progress ongoing...}

\end{document}